\documentclass[12pt]{article}

\usepackage[english]{babel}
\usepackage[top=3cm,bottom=3cm,left=3cm,right=3cm]{geometry}

\linespread{1.25} % Line spacing.
\usepackage{graphicx} %To insert images.
%\usepackage[skip=10pt plus1pt, indent=20pt]{parskip} | Space after each paragraph.


\usepackage{xcolor} %Color: use command of the form \textcolor{color}{text}.

\usepackage{amssymb} 
\usepackage{amsmath} %Math symbols.

\usepackage{apacite}
\bibliographystyle{apacite} %Bibliography.

\usepackage{lipsum} %Dummy text.

\newcommand{\nameddefinition}[1]{\textbf{#1}} % Define a custom command for non-numbered theorem-like definitions.

\title{On the Value of Philosophy}
\author{Franco Menares-Paredes}
\date{\today}

\begin{document}
\maketitle

\section{Introduction}

%Introduce the issue through a story. Hopefully, it could illustrate the following points. (i) Difference between "internal" and "external" justifications of the existence of the discipline. (ii) The fact that philosophy receives public funding generates a duty on us of having a good external justification. (iii) Ppl from humanities often make bad external justifications. (iv)(supererogatory) The value-of-philosophy question is more important than the progress-in-philosophy question.
Carla is the rector of a university. She is a philosopher by training and has the project of building a strong philosophy department. She proposes to the board to inject more resources into the department. The board is skeptical. Applications to philosophy programs are relatively low, which suggest that there is not much interest in the field. Shouldn't the university favor programs that attract the interest of more students? Also, the most talented students seem to be more interested in other fields, such as mathematics, physics, or biology. The board see no reason to support Carla's project of building a strong philosophy department. Carla has some influence among politicians. She periodically writes columns in important newspapers and people in positions of power sometimes take note of her opinions and some times even act on them. So Carla decides to go one step forward and champion the project of a generalized injection of founds to develop the field in her country—maybe through .... She decides to endeavor the task of convincing the government to support her project. She has to present a case to a board of police makers. Many of them respect her for her work as a public intellectual, and some of them are actually enthusiastic consumers of philosophy. They would be keen to spend public funds on Carla's project, but the are not sure they can justify it to the public. Not too long ago there was an affair after the same board cut funds for the training of medical specialists. They are afraid of the reaction of the public to such an announcement. Can they justify raising the funding for the development of philosophy while cutting funds for the training of medical specialists? 

%Motivation to adress the value-of-philosophy question instead of the progress-in-philosophy one.
There is a nice discussion in contemporary philosophy about progress in philosophy. Motivated by this literature I want to adress a closely related question: what makes philosophy a field worth existing? what justifies its place among other human practices such as mathematics, physics, or biology? In short, what is—what grounds—the value of philosophy?

%Teaser of the main argument.

%The structure of the paper.

\newpage

\bibliography{bibliography}

\end{document}